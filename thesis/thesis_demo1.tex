% preamble lines to specify the formatting
\documentclass[12pt, a4paper]{article}
% page margin left=3cm, right=4m
% you could change it to left=1.5in, right=1in, top=1in, bottom=1in depends on
% the specification of your professor
\usepackage[left=3cm, right=4cm, top=1in, bottom=1in]{geometry}
\linespread{1.25}  
% standard line skip means a factor of 1.2 (such as font height 10pt, base line skip 12pt).
% Multiply with \linespread, so you get 1.25*1.2 = 1.5, so one-half.
\usepackage[T1]{fontenc}  % allows the user to input accented characters directly from the keyboard;
\usepackage[utf8]{inputenc}  % enable fonts to use for printing characters.
\usepackage[english]{babel}  % languages 
\usepackage{mathptmx}  % use Times as default text fault
\usepackage{authblk}  % enable author affiliation
\usepackage{tikz}  % graph drawing and position
\usepackage{graphicx}  % graph drawing and position
\usepackage{tikzpagenodes}  % add nodes and notation on graph
\usepackage{wrapfig}  % wrap figures
\usepackage[export]{adjustbox}  % graph align
\usepackage{blindtext}  % generate blind text
\usepackage[page,toc,titletoc,title]{appendix}  % appendix, table of contents
\usepackage{listings}  % code highlight
\usepackage{natbib}  % manage citation
% A parameter that allows \hbox's to be overfull by [length] before an overfull error occurs.
\hfuzz=100pt  % suppress the \hbox full error
\usepackage{microtype} % Improves typography



% \renewcommand{\headrulewidth}{0pt}
% \fancyhead[L]{}
% \fancyhead[R]{\includegraphics[width=4cm]{Logo.pdf}}
%----------------------------------------------------------------------
% where the document starts
%-----------------------------------------------------------------------
\begin{document}


%%%------------------------Cover page-----------------------------%%%
% University logo
\begin{titlepage}
    \makebox[1.1\textwidth][r]{\includegraphics[width=.29\textwidth,right]{./logos/logobw.png}}\par\vspace{1.5cm}
    \centering
	{\LARGE\bfseries Thesis or Report Title, Like A Cross-country Analysis of Innovation \par}
	\vspace{0.7cm}
    % if you don't italic shape, change delete \itshape
	{\Large\itshape John Birdwatch, (Maybe the second author)\par} 
    \vspace{0.7cm}
    % if you want to make it more specific
    % {Presented to Professorship of Management and Microeconomics \par}
    % {Faculty of Economics and Business}
    {Presented to Faculty of Economics and Business \par}
    {in partial fulfillment of the requirements\par}
    {for the degree of B.A. in Economics (or for the course/seminar)\par}
    \vspace{0.3cm}
    {\large Goethe-Universität Frankfurt am Main \par}
    \vspace{2cm}
	supervised by\par
    \textsc{Prof. Dr. Cornelia Storz} \par
    Fei Wang \par 
    \vspace{3cm}
    \begin{flushleft}  % flushright  if you like
    \scshape\small Author Information \par
    Matriculation Number: 497516 \par 
    Email: youremail@domain.com \par 
    Street Love 101, 60323 Frankfurt am Main \par
    Semester: Summer \par 
    Extra information (like seminar organizer, etc.)
    \end{flushleft}

	\vfill   % fill the space

% Bottom of the page
	{\large July 6, 2020\par}  
\end{titlepage}

%%%------------------------Abstract-----------------------------%%%
% delete this part if you don't need a section of abstract
\newpage
\thispagestyle{empty}
\begin{abstract}
    An abstract is a concise summary of a completed research project or paper. 
    A well-written abstract will make the reader want to learn more about your research, 
    read your paper, or attend your presentation. Abstracts also serve as a summary 
    of the research so the paper can be categorized and searched by subject and keywords.
\end{abstract}
%%%------------------------End of Abstract-----------------------%%%

%%%------------------------Table of contents----------------------%%%
% you don't need to change anything for this part
\newpage
\pagenumbering{roman}
\setcounter{tocdepth}{2}
\tableofcontents
\listoffigures
\listoftables
%%%---------------------------End of TOC--------------------------%%%



%%%-----------------MAIN BODY: now, you can enjoy your writing--------------%%%
\newpage
\pagenumbering{arabic}
\setcounter{page}{1}
\section{Introduction}

This LaTex template is for students who are taking courses from \textit{Chair
for the Study of Economic Institutions, Innovation, and East Asian Development}
at Goethe University in Frankfurt. The Top and bottom margins are one inch. The left
margin is 3cm, and the right is 4cm. It uses Times New Roman font and 12pt size.
The text is justified with 1.5 line spacing. 

The main advantage of using LaTex is the efficiency of citation. When you write
any academic paper, you need to cite sources properly. In \texttt{WORD}, you have
to copy the right references in the right format and organize them in alphabetical
order. With LaTex, all you need to do is to type:
\begin{lstlisting}
    \cite{}  % gives inline citation
    \citep{} % gives Parenthetical citation
\end{lstlisting}

The above code would generate the citation and reference entry for you automatically.
For instance, according to \cite{christensen2013disruptive} disruptive innovation
refers to innovations and technologies that make expensive or sophisticated 
products and services accessible and more affordable to a broader market \citep{greenwade93}.

Every time you cite, the reference will be added on the reference list.



\section{Literature Review}

Write your literature review here. You can use
AI to help you write the literature review. For instance, you can use


\subsection{Nature of Innovation}

\subsection{Innovation Strategies}

\section{Data Description}

\blindtext

\section{Methodology}


\section{Results and Discussion}

\section{Conclusion}






\newpage
\bibliography{eco}
\bibliographystyle{apalike}





\end{document}